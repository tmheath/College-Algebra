\documentclass{amsart}

\usepackage[margin=0.5in]{geometry}
\usepackage{amssymb}

\title{College Algebra Section 0.10 Homework}
\author{Timothy Heath}
\date{\today}

\newtheorem{question}{Question}

\begin{document}
	\maketitle
	\newpage
	\begin{question}
		Find the complex conjugates of the following numbers:\\
		\textbf{a.} $\overline{11}=\boxed{11}$\\
		\textbf{b.} $\overline{20i}=\boxed{-20i}$\\
		\textbf{c.} $\overline{-95+88i}=\boxed{-95-88i}$\\
		\textbf{d.} $\overline{-483i-84}=\boxed{-84+483i}$\\
		\textbf{a.}\\
		\[
		\begin{aligned}
			\overline{11}\\
			\overline{11+0i}\\
			11-0i\\
			11\\
		\end{aligned}
		\]
		\textbf{b.}\\
		\[
		\begin{aligned}
			\overline{20i}\\
			\overline{0+20i}\\
			0-20i\\
			-20i\\
		\end{aligned}
		\]
		\textbf{c.}\\
		\[
		\begin{aligned}
			\overline{-95+88i}\\
			-95-88i\\
		\end{aligned}
		\]
		\textbf{d.}\\
		\[
		\begin{aligned}
			\overline{-483i-84}\\
			\overline{-84-483i}\\
			-84+483i\\
		\end{aligned}
		\]
	\end{question}
	\begin{question}
		You find the discriminant D of a quadratic equation
		and notice that $D=0$. What does this tell you about
		the solution(s) to the equation?\\
		\textbf{a.} The equation has one real solution.\\
		\textbf{b.} The equation has two real number solutions.\\
		\textbf{c.} The equation has two complex number solutions.\\
		\textbf{d.} The equation has one real and one complex solution.\\
		\textbf{The answer is a.}\\
		\[\begin{aligned}
			D&=0\\
			x&=\frac{-b\pm\sqrt{b^2-4ac}}{2a}\\
			x&=\frac{-b\pm\sqrt{D}}{2a}\\
			x&=\frac{-b\pm\sqrt{0}}{2a}\\
			x&=\frac{-b\pm0}{2a}\\
			x&=\frac{-b}{2a}\\
		\end{aligned}\]
	\end{question}
	\begin{question}
		Add.\\
		$(-1-6i)+(6+10i)=\boxed{5+4i}$\\
		\[
		\begin{aligned}
			(-1-6i)+(6+10i)&=\\
			-1-6i+6+10i&=\\
			6-1+10i-6i&=\\
			5+4i&=\\
		\end{aligned}
		\]
	\end{question}
	\newpage
	\begin{question}
		Perform the indicated operations and simplify.\\
		Add: $(13-25i)+(6+7i)$\\
		sum $=\boxed{19-18i}$\\
		Subtract: $(13-25i)-(6+7i)$\\
		difference $=\boxed{7-32i}$
		\[\begin{aligned}
			(13-25i)+(6+7i)&=\\
			13+6-25i+7i&=\\
			19-18i&=\\
			(13-25i)-(6+7i)&=\\
			13-6-25i-7i&=\\
			7-32i&=\\
		\end{aligned}\]
	\end{question}
	\begin{question}
		Simplify, write in the for a+bi.\\
		$(2+11i)(4-8i)=\boxed{96+28i}$
		\[\begin{aligned}
			(2+11i)(4-8i)&=\\
			8-16i+44i-88i^2&=\\
			8+28i+88&=\\
			96+28i&=\\
		\end{aligned}\]
	\end{question}
	\begin{question}
		Perform the indicated operation and simplify.
		Express the answer as a complex number.\\
		$(11-12i)(-2+9i)=\boxed{86+123i}$
		\[\begin{aligned}
			(11-12i)(-2+9i)&=\\
			-22+99i+24i-108i^2&=\\
			-22+123i+108&=\\
			86+123i&=\\
		\end{aligned}\]
	\end{question}
	\begin{question}
		Multiply the following complex number by it's conjugate, and simplify: $(1-5i)$
		$$\boxed{26}$$
		\[\begin{aligned}
			(1-5i)&\\
			(1-5i)&(1+5i)\\
			1-&25i^2\\
			1+&25\\
			2&6\\
		\end{aligned}\]
	\end{question}
	\begin{question}
		 In this problem you are going to investigate what happens when you multiply a complex number by its conjugate using the number $1-i$.\\
		  First, multiply it by something that is not its conjugate: \\
		  $(1-i)(-9-5i)=\boxed{-14+4i}$\\
		  Now, multiply it by its conjugate: \\
		  $(1-i)(1+i)=\boxed{2}$\\
		  You should notice a difference in those two results. To test it, try another one: \\
		  $(-5+4i)(-5-4i)=\boxed{41}$\\
		  When you multiply a complex number by it's conjugate
		  the result will be a real number.
	\end{question}
	\begin{question}
		\[\begin{aligned}
			(-1+&4i)^2\\
			1-8&i-16\\
			-15&-8i\\
		\end{aligned}\]
	\end{question}
	\newpage
	\begin{question}
		Let $f(x)=x^2+x-4$.\\
		$f(3+i)=\boxed{7+7i}$\\
		$f(-i)=\boxed{-5-i}$\\
		\[\begin{aligned}
			f(3+i)&=(3+i)^2+(3+i)-4\\
			&=9+6i-1+3+i-4\\
			&=7+7i\\
			f(-i)&=(-i)^2+(-i)-4\\
			&=-1-i-4\\
			&=-5-i
		\end{aligned}\]
	\end{question}
	\begin{question}
		\[i^{14}=-1\]
	\end{question}
	\begin{question}
		\[i^{71}+i^{72}+i^{73}=1\]
	\end{question}
	\begin{question}
		\[\frac{-2+2i}{6i}=\frac{1}{3}+\frac{1}{3}i\]
	\end{question}
	\begin{question}
		\[\frac{-1}{2i}=\frac{i}{2}\]
	\end{question}
	\begin{question}
		\[\frac{3}{4+i}=\frac{12-3i}{17}\]
	\end{question}
	\begin{question}
		\[\frac{i}{3-2i}=\frac{-2+3i}{13}\]
	\end{question}
	\begin{question}
		Find all complex solutions of the equation
		$x^2+8x+41=0$.\\
		$x=\boxed{-4+5i,-4-5i}$
		\[\begin{aligned}
			x^2+8x+41&=0\\
			x^2+8x+16+25&=0\\
			(x+4)^2+25&=0\\
			(x+4)^2+5^2&=0\\
			(x+4)^2&=-5^2\\
			x+4&=\pm\sqrt{-5^2}\\
			x+4&=\pm\sqrt{-25}\\
			x+4&=\pm5i\\
			x&=-4\pm5i\\
		\end{aligned}\]
	\end{question}
	\begin{question}
		Solve via completing the square.
		\[\begin{aligned}
			p^2-12p+17&=49\\
			p^2-12p+36&=68\\
			(p-6)^2&=68\\% First box, got super lazy... bad eye strain hurts.
			p-6&=\pm2\sqrt{17}\\
			p&=6\pm2\sqrt{17}\\
		\end{aligned}\]
	\end{question}
	\begin{question}
		\[\begin{aligned}
			w^2-8w+36&=-60\\
			w^2-8w+16&=-80\\
			(w-4)^2&=-80\\
			w-4&=\pm\sqrt{-80}\\
			w-4&=\pm4i\sqrt{5}\\
			w&=4\pm4i\sqrt{5}\\
			m^2+4m-38&=57\\
			m^2+4m+4&=99\\
			(m+2)^2&=99\\
			m+2&=\pm\sqrt{99}\\
			m+2&=\pm3\sqrt{11}\\
			m&=-2\pm3\sqrt{11}\\
		\end{aligned}\]
	\end{question}
	\begin{question}
		\[\begin{aligned}
			x^4-5x^2-6&=0\\
			(x^2)^2-5x^2-6&=0\\
			(x^2-6)(x^2+1)&=0\\
			x^2-6=0&\text{, or }x^2+1=0\\
			(x-\sqrt{6})(x+\sqrt{6})=0&\text{, or }(x+i)(x-i)=0
			\\
			\therefore x &= (-\sqrt{6}, +\sqrt{6}, -i, i)
		\end{aligned}\]
	\end{question}
\end{document}
