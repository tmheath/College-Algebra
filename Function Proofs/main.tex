\documentclass{amsart}

\usepackage[margin=0.5in]{geometry}
\usepackage{amssymb}

\title{Proof for proving an equality a function | College Algebra}
\author{Timothy Heath}
\date{\today}

\newtheorem{theo}{Theorem}
\newtheorem{lem}{lemma}
\newtheorem{prop}{Proposition}
\newtheorem{exa}{Example}
\newtheorem{defn}{Definition}

\begin{document}
	\maketitle
	\newpage
	\begin{defn}[Relation]
		A relation shows how two quantities
		relate to each other. A relation may
		be either an equality or an inequality.
		\begin{exa}
			$$x=1$$
		\end{exa}
		\begin{exa}
			$$1=1$$
		\end{exa}
		\begin{exa}
			$$x+5=1$$
		\end{exa}
		\begin{exa}
			$$x>1$$
		\end{exa}
		\begin{exa}
			$$4x+3<1$$
		\end{exa}
	\end{defn}
	\begin{defn}[Equation]
		An Equation is just a relation showing
		one quantity being the same as some
		other quantity.
		\begin{exa}
			$$4x^2+3x=7$$
		\end{exa}
	\end{defn}
	\begin{defn}[Inequality]
		An Inequality shows that two quantities are not the
		same.
		\begin{exa}
			$$4x^2+3x\neq7$$
		\end{exa}
	\end{defn}
	\begin{defn}[Set]
		A set is a collection of unique objects.
		\begin{exa}
			$$\{1,2,3\}$$
		\end{exa}
		\begin{exa}
			$$\{1,2,x\}$$
		\end{exa}
		\begin{exa}
			$$\{1,2,\{1,2,x\}\}$$
		\end{exa}
		\begin{exa}
			$$\{1,1,2\}=\{1,2\}$$
		\end{exa}
		\begin{exa}
			$$(-\infty,\infty)=\mathbb{R}$$
		\end{exa}
	\end{defn}
	\begin{defn}[Expressed relation as set | graph data]
		Any relation may be expressed as a set of points.
		\begin{exa}
			The set of all points from the relation.
			$$\{(x,y)|x^2=y\}$$
		\end{exa}
	\end{defn}
	\begin{defn}[Domain]
		A domain of a relation shows the allowed input values.
		\begin{exa}
			Let $A$ be $\{(x,y)|x^2=y\}$.\\
			Then the domain of A is...\\
			$$D_A=\{x|(x,y)\in A\}$$
		\end{exa}
	\end{defn}
	\begin{defn}[Range]
		A range of a relation shows the possible output
		values.
		\begin{exa}
			Let $A$ be $\{(x,y)|x^2=y\}$.\\
			Then the range of A is...\\
			$$R_A=\{y|(x,y)\in A\}$$
		\end{exa}
	\end{defn}
	\begin{defn}[Function]
		A function is an equality where no two separate
		points share the same x value.
		\begin{exa}
			$$f(x)=x^2$$
		\end{exa}
	\end{defn}
	To prove an equality is a function you must demonstrate
	that there is at least a single x value that produces
	two solutions. I will demonstrate here that instead of
	proving directly, one may simply view the domain of an
	equality to determine whether or not it is a function.
	\begin{proof}
		$$\text{The equality}$$
		$$\{(x,y)|x^2=y\}$$
		$$\text{In order to prove the equality to be a
		function, we must demonstrate that there are at
		least two points with the same y value.}$$
		$$\text{If there are two x values that give the
		same y value, then there are more points then there
		are x values.}$$
		$$\text{From a high level this may be shown via:}$$
		$$\text{Let $A$ be }\{(x,y)|x^2=y\}$$
		$$\text{If the relation is a function then:}$$
		$$n(D_A)=n(A)$$
		$$\therefore\text{In order to prove an equality is
		a function, you only need to demonstrate that there
		are more points than there are x values.}$$
		$$\text{In order to prove there are more points
		than x values, you must show that at least one
		x value gives two points.}$$
		\end{proof}
\end{document}
